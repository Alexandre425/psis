\documentclass[a4paper]{report}

\usepackage[a4paper, total={6in, 9in}]{geometry}
\usepackage[table,xcdraw,svgnames]{xcolor}
\usepackage{indentfirst}
\usepackage{svg}
\usepackage{float}
\usepackage{mathtools}
\usepackage{dirtree}
\usepackage{graphicx}
\usepackage{multirow}
\usepackage{listings}
\usepackage{titlesec}
\usepackage{hyperref}

\titleformat{\chapter}[display]   
{\normalfont\huge\bfseries}{\chaptertitlename\ \thechapter}{20pt}{\Huge}   
\titlespacing*{\chapter}{0pt}{0pt}{10pt}

\graphicspath{ {./img/} }
\renewcommand{\chaptername}{Parte}
\renewcommand{\contentsname}{Índice}
\renewcommand{\figurename}{Figura}
\renewcommand{\tablename}{Tabela}

\definecolor{mauve}{rgb}{0.88, 0.69, 1.0}
\lstset{
    frame=tb,
    language=C,
    aboveskip=3mm,
    belowskip=3mm,
    showstringspaces=false,
    columns=flexible,
    basicstyle={\small\ttfamily},
    numbers=none,
    numberstyle=\tiny\color{gray},
    keywordstyle=\color{blue},
    commentstyle=\color{DarkGreen},
    stringstyle=\color{mauve},
    breaklines=true,
    breakatwhitespace=true,
    tabsize=4
}

\setlength{\parskip}{3pt}

\begin{document}

\begin{titlepage}
    \begin{center}
        \vspace*{3cm}
 
        \LARGE
        \textbf{Instituto Superior Técnico}
        \vskip 0.4cm
 
        \Large{MEEC}
        \vskip 0.2cm

        \Large{Programação de Sistemas}
        \vskip 3cm
        

 
        \Huge{\textbf{Projeto}}
        \vskip 0.5cm

        \huge{\textbf{Multiplayer Pacman}}
        \vskip 0.5cm

        \vfill
 
        \large
        \vspace{0.3cm}
        Alexandre Rodrigues, 90002\\
        \vspace{1cm}
    \end{center}
\end{titlepage}

\tableofcontents

\setcounter{chapter}{1}
\chapter*{}
    \section{Arquitetura}
        \subsection{Nós}
            \par Este projeto implementa uma arquitetura servidor/cliente simples. O servidor aguarda a conexão dos clientes, comunicando com estes de modo a receber e aplicar o \textit{input} do jogador, e enviar o resultante estado do jogo.

        \subsection{Módulos}
            \par O cliente e o servidor estão ambos divididos em três módulos, sendo estes o módulo principal de jogo, o módulo de mensagem e o de conexão. 
            \par O primeiro dos três implementa a lógica que rege o jogo \textit{Pacman}, como as interações entre os personagens e \textit{delays} nas ações.
            \par O módulo de mensagem implementa funções que definem o protocolo de comunicação entre o cliente e o servidor. Por exemplo, enquanto o módulo de mensagem do servidor implementa a função \textit{message\_send\_board}, o módulo do cliente implementa a função análoga \textit{message\_recv\_board}.
            \par Por último, o módulo de conexão lida, como seria de esperar, com a conexão entre o cliente e o servidor. No cliente, por exemplo, implementa a função que faz a ligação inicial ao servidor e a que recebe as mensagens do servidor. No lado do servidor trata ainda de guardar informação relevante aos clientes atualmente conectados.
        
        \subsection{Threads}
            \par Enquanto que o cliente tem apenas duas \textit{threads}, o servidor tem uma \textit{thread} principal e tantas outras quantos os clientes que estiverem conectados.
            \par O cliente tem uma \textit{thread} responsável por desenhar a \textit{board} e o estado do jogo em geral, bem como adquirir o \textit{input} do jogador e enviá-lo para o servidor, e uma \textit{thread} responsável por receber mensagens do servidor (\textit{recv\_from\_server}), atualizando o estado do jogo comforme o necessário.
            \par A \textit{thread} principal do servidor processa o jogo em si, alterando o estado deste consoante os movimentos enviados pelos clientes, e enviando o resultado para os clientes. Cada cliente que se conecta causa a criação de uma \textit{thread} que recebe exclusivamente as mensagens do respetivo cliente (\textit{recv\_from\_client}), atualizando a informação utilizada pela \textit{thread} principal.

    \section{Organização do código}
        \par O código está dividido em três categorias principais, sendo estas código que diz respeito ao cliente, ao servidor, e a ambos. A diretoria do projeto está organizada da forma seguinte:

        \begin{figure}[H]
            \dirtree{%
                .1 Project.
                .2 bin.
                .2 board-drawer.
                .2 res.
                .2 src.
                .3 common.
                .4 message.c.
                .4 message.h.\
                .4 pacman.c.
                .4 pacman.h.
                .4 UI\_library.c.
                .4 UI\_library.h.
                .4 utilities.c.
                .4 utilities.h.
                .4 vector.c.
                .4 vector.h.
                .3 client.
                .4 client\_connection.c.
                .4 client\_connection.h.
                .4 client\_message.c.
                .4 client\_message.h.
                .4 client.c.
                .4 client.h.
                .3 server.
                .4 server\_connection.c.
                .4 server\_connection.h.
                .4 server\_message.c.
                .4 server\_message.h.
                .4 server.c.
                .4 server.h.
            }
            \label{}
        \end{figure}

        \par As diretorias \textit{client} e \textit{server} têm o código que diz respeito ao cliente e ao servidor, respetivamente. Cada uma delas implementa os respetivos módulos. 
        \par Alguns destes módulos dependem de funções, estruturas ou definições da diretoria \textit{common}. Por exemplo, \textit{message.h} define \textit{MessageTypes}, que os módulos de mensagem utilizam para identificar e processar corretamente mensagens entre eles. \textit{pacman.c} implementa a \textit{board} do jogo, usada igualmente pelo cliente e servidor. \textit{vector.c} e \textit{utilities.c} implementam funções mais genéricas, como operações vetoriais, alocação de memória com verificação, ou conversões de formato.
        \par O código está comentado nos ficheiros \textit{header}, cada função sendo sucintamente explicada pelo comentário que a antecede. Os ficheiros de código também têm comentários ao longo das funções e estruturas quando código não é evidente. De qualquer dos modos, o código foi escrito de modo explicito e verboso, quer seja no nome das funções ou variáveis, com o intuito de facilitar a sua leitura.

    \section{Estruturas de dados}
        \par O projeto possui algumas estruturas de dados comuns a ambos os lados servidor e cliente, bem como algumas que são análogas mas diferentes, devido aos requesitos diferentes do cliente e do servidor. Por exemplo, observando a estrutura \textit{Player}, que guarda informação relevante a um jogador no contexto do jogo, do lado do cliente:
        \begin{lstlisting}
            typedef struct _Player
            {
                unsigned int player_id;
                Color color;
                unsigned int score;
                int powered_up;
                Vector* pacman_pos;
                Vector* monster_pos;
            } Player;
        \end{lstlisting}
        \par Observamos que esta é diferente da estrutura análoga do lado do servidor:
        \begin{lstlisting}
            typedef struct _Player
            {
                unsigned int player_id;
                Color color;
                int powered_up;                         // Stores 0, 1 or 2, depending on how many monsters the pacman can still eat
                unsigned int score;                     // The number of things eaten
                Vector* pacman_pos;
                char pacman_move_dir;                   // Movement direction
                struct timespec pacman_last_move_time;  // Time of last movement
                Vector* monster_pos;
                char monster_move_dir;
                struct timespec monster_last_move_time;
            } Player;
        \end{lstlisting}
        \par Como seria de esperar, o servidor tem uma implementação mais complexa da mesma estrutura, pois é este que é responsável pelo processamento dos dados. Guarda a mais a direção de movimento do jogador, bem como o tempo em que o último movimento foi executado, para ambos os personagens.

        \par A estrutura \textit{Game} é o método principal de passar informação às diversas funções (de modo a evitar variáveis globais), contendo toda a informação que diz respeito ao jogo. Mais uma vez existem discrepâncias na sua implementação dado que no lado do cliente:
        \begin{lstlisting}
            typedef struct _Game
            {
                int server_socket;
                unsigned int player_id;
                Board* board;
                unsigned int n_players;
                Player** players;
                unsigned int n_fruits;
                Fruit* fruits;
            } Game;
        \end{lstlisting}
        \par E no lado do servidor:
        \begin{lstlisting}
            typedef struct _Game
            {
                Board* board;
                unsigned int max_players;
                unsigned int n_players;
                Player** players;
                unsigned int n_fruits;
                Fruit** fruits;
            } Game;
        \end{lstlisting}
        \par O cliente necessita de saber o seu \textit{player\_id} e, por conveniência, dado que o cliente só comunica com uma outra entidade, a socket do servidor. Já o servidor guarda também o número máximo de jogadores, que o cliente não precisa de saber.

        \par O servidor tem uma estrutura exclusiva que armazena informação relevante à conexão a um cliente, \textit{Client}, em \textit{server\_connection.c}:
        \begin{lstlisting}
            typedef struct _client
            {
                Game* game;
                pthread_t handler_thread;
                unsigned int player_id;
                int socket;
            } Client;
        \end{lstlisting}

        \par Quanto às frutas, estas são implementadas \textit{client-side} por:
        \begin{lstlisting}
            typedef struct _Fruit
            {
                unsigned int fruit_type;    // Cherry or lemon
                int is_alive;               // Is it on the board? (Or waiting to respawn)
                Vector* pos;
            } Fruit;
        \end{lstlisting}
        \par E do lado do servidor:
        \begin{lstlisting}
            typedef struct _Fruit
            {
                unsigned int fruit_type;    // Cherry or lemon
                int is_alive;               // Is it on the board? (Or waiting to respawn)
                struct timespec eaten_time; // Time it was last eaten
                Vector* pos;
            } Fruit;
        \end{lstlisting}
        \par Mais uma vez, como o servidor implementa o jogo em si, precisa de guardar o instante em que a fruta é comida para fazer o seu \textit{respawn}, algo que não acontece no cliente.

        \par A \textit{board} do jogo é implementada de forma sucinta e privada:
        \begin{lstlisting}
            // The game board
            typedef struct _Board
            {
                Vector* board_size;
                unsigned int** board;
            } Board;
        \end{lstlisting}

        \par Por fim, os vetores são também definidos, simples e privadamente por:
        \begin{lstlisting}
            typedef struct _Vec
            {
                int x;
                int y;
            } Vector;
        \end{lstlisting}
    
    \section{Protocolo de comunicação}

        \par A comunicação entre o cliente e o servidor é feita através de mensagens discretas, identificadas pelo seu \textit{MessageType}:
        \begin{lstlisting}
            typedef uint16_t MessageType;
            enum _MessageType {
                // Terminates a message, used for checking allignment
                MESSAGE_TERMINATOR = UINT16_MAX, 
                // Player color
                MESSAGE_COLOR = 0, 
                // The game board
                MESSAGE_BOARD,
                // ...
                // Resto das mensagens em common/message.h
                // ...
                // Alerts the server is full
                MESSAGE_SERVER_FULL
                };
        \end{lstlisting}

        \par \textit{message.c} também implementa diversas funções de envio e recebimento de dados de tipos específicos e explicitos quanto ao tamanho e sinal. Tem-se por exemplo:
        \begin{lstlisting}
            int message_send_uint16_t(int socket, uint16_t message)
            {
                uint16_t m_net = htons(message);
                return send_all(socket, (void*)&m_net, sizeof(uint16_t));
            }
            int message_recv_uint16_t(int socket, uint16_t* message)
            {
                int ret = recv_all(socket, (void*)message, sizeof(uint16_t));
                *message = (uint16_t)ntohs(*message);
                return ret;
            }
        \end{lstlisting}

        \par A partir destas funções, são implementadas funções mais abstratas nos módulos de comunicação específicos ao cliente ou servidor. Tomando o exemplo simples (o qual é seguido pela maioria das funções do mesmo tipo) da função (do lado do servidor, em \textit{server\_message.c}) que envia o \textit{player\_id} ao seu cliente respetivo:
        \begin{lstlisting}
            void message_send_player_id(int socket, unsigned int player_id)
            {
                message_send_uint16_t(socket, (uint16_t)MESSAGE_PLAYER_ID);
                message_send_uint32_t(socket, (uint32_t)player_id);
                message_send_uint16_t(socket, (uint16_t)MESSAGE_TERMINATOR);
            }
        \end{lstlisting}
        \par E a sua análoga encarregue do recebimento (no cliente, em \textit{client\_message.c}):
        \begin{lstlisting}
            void message_recv_player_id(int socket, unsigned int* player_id)
            {
                message_recv_uint32_t(socket, (uint32_t*)player_id);
            }
        \end{lstlisting}
        \par Como se observa, a função de envio envia não só os dados. Envia primeiro o \textit{MessageType}, de modo a que o cliente possa identificar o conteúdo, seguido dos dados da mensagem, e finalmente um terminador de mensagem. Este último permite já fazer uma validação preliminar dos dados quanto ao seu comprimento como se poderá observar de seguida.
        
        \par A função de recebimento está encarregue exclusivamente de receber os dados. Isto porque a identificação e verificação do alinhamento da mensagem é feito em \textit{recv\_from\_server} (em \textit{client\_connection.c}), uma versão resumida da qual está abaixo:
        \begin{lstlisting}
            void* recv_from_server(void* _game)
            {
                // Cast to game
                Game* game = (Game*)_game;
                int server_socket = game_get_server_socket(game);

                // Probe for new messages
                while (1)
                {
                    // Determine message type
                    MessageType mt;
                    int ret = message_recv_uint16_t(server_socket, (uint16_t*)&mt);
                    // ...
                    switch (mt)
                    {
                    // ...
                    case MESSAGE_PLAYER_ID:
                    {
                        unsigned int player_id;
                        message_recv_player_id(server_socket, &player_id);
                        game_set_player_id(game, player_id);
                        break;
                    }
                    // ...
                    default:
                        break;
                    }

                    // Receive the terminator
                    message_recv_uint16_t(server_socket, (uint16_t*)&mt);
                    if (mt != MESSAGE_TERMINATOR)
                        message_misaligned();
                }
                client_quit();
                return NULL;
            }
        \end{lstlisting}
        \par A função recebe o tipo de mensagem, escolhendo através do \textit{switch} a função adequada para interpretar os dados que se seguem. Depois da mensagem ser lida verifica o alinhamento e, caso não esteja alinhada, o cliente interrompe a execução. Na prática, nunca há um desalinhamento dos dados, sendo que isto foi implementado de modo a ajudar com o densenvolvimento das diversas mensagens.
        \par A função \textit{recv\_from\_client} do servidor é quase idêntica, bem como todo o processo detalhado anteriormente. No entanto, neste contexto não faria sentido interromper a execução caso exista desalinhamento de mensagens. Se isso fosse feito, qualquer cliente malicioso poderia enviar um pacote de dados demasiado longo ou curto e parar o servidor. Neste caso, um desalinhamento dos dados resulta então na terminação da conexão com o cliente que o causou. 
        \par Há que salientar que um desalinhamento nunca ocorre em situações normais, pois as mensagens foram desenhadas de forma a serem discretas e de tamanho explicito, sendo que não há risco do uma das partes ler menos ou mais informação do que o que a mensagem contém.

        \par O servidor tem ainda outra forma de comunicação única, a qual utiliza para atualizar os clientes com o estado do jogo. A função \textit{send\_to\_all\_clients} envia uma mensagem idêntica para todos os clientes atualmente conectados a comando da \textit{thread} principal. Um resumo dessa função encontra-se abaixo:
        \begin{lstlisting}
            void send_to_all_clients(Game* game, MessageType message_type, void* extra_data)
            {
                pthread_mutex_lock(&client_array_lock);
                // For every client
                for (unsigned int i = 0; i < n_clients; ++i)
                {
                    // Send the message defined by message_type
                    switch (message_type)
                    {
                    // ...
                    case MESSAGE_PLAYER_LIST:
                        message_send_player_list(client_array[i]->socket, game);
                        break;
                    case MESSAGE_PLAYER_DISCONNECT:
                    {
                        unsigned int* player_id = (unsigned int*)extra_data;
                        message_send_player_disconnect(client_array[i]->socket, *player_id);
                        break;
                    }
                    // ...
                    default:
                        break;
                    }
                }
                pthread_mutex_unlock(&client_array_lock);
            }
        \end{lstlisting}

    \section{Validação de dados}
        \par Na transação de informação, é especialmente importante protejer o servidor de informação possívelmente prejudicial. Deste modo, é necessário verificar que os dados que chegam ao servidor são válidos, e agir de acordo com o resultado. Um exemplo de verificação de dados está representado abaixo, um excerto da função \textit{receive\_from\_client} (funcionalmente idêntica à analisada acima) do ficheiro \textit{server\_connection.c}:
        \begin{lstlisting}
            case MESSAGE_MOVE_PAC:
            {
                char move_dir;
                message_recv_movement(client->socket, (char*)&move_dir);
                if (move_dir == 'w' || move_dir == 'a' || move_dir == 's' || move_dir == 'd' || move_dir == (char)0)
                    player_set_pac_move_dir(player, move_dir);
                break;
            }
        \end{lstlisting}
        \par Caso um cliente enviasse uma direção de movimento inválida, poderiam haver consequências graves para o funcionamento do servidor, a não ser que o movimento também fosse verificado mais à frente, antes dos personagens serem movidos, por exemplo. Ao verificar assim que a mensagem é recebida, e ignorando-a caso não seja válida, assegura-se a proteção do servidor de dados inválidos e simplifica-se o processo de \textit{error-checking}.

        \par Outro exemplo de leitura e interpretação de erros está relacionado com o método escolhido para terminar a \textit{thread} que aceita conexões dos novos clientes. Quando o servidor fecha por meio da janela \textit{SDL2}, há que alertar as outras \textit{threads} do processo. Isto é feito por meio de um sinal enviado através de \textit{pthread\_kill}, de modo a que o \textit{accept} (na \textit{thread} que gere as conexões) ou o \textit{read} (nas \textit{threads} que comunicam com os clientes) desbloqueiem. Após isto, estas funções retornam -1, mas é importante analisar mais profundamente (através da variável \textit{errno}) o erro, determinando se se trata da interrupção ou de outro erro indesejado. Um excerto relevante da função \textit{connect\_to\_clients} de \textit{server\_connection.c} encontra-se abaixo:
        
        \begin{lstlisting}
            // ...
            // Tell the kernel to listen on this socket
            listen(listen_socket, 5);

            // Accept all incoming connections
            int client_socket;
            while (1)
            {
                // Accept connections - blocks until a client connects
                client_socket = accept(listen_socket, NULL, NULL);
                if (client_socket == -1)
                {
                    // If accept was interrupted by a signal (means server is shutting down)
                    if (errno == EINTR)
                    {
                        break;
                    }
                    else
                    {
                        perror("ERROR - Accept failed");
                        exit(EXIT_FAILURE);
                    }
                }
                // ...
        \end{lstlisting}

    \section{Funcionalidades Implementadas}

        \subsection{Controlo dos personagens pelo cliente}
            \par O cliente controla o Pacman segurando o botão esquerdo do rato e movendo o ponteiro de acordo com a direção que se quer deslocar (se quiser que o Pacman se desloque para cima, coloca o ponteiro acima do Pacman). O monstro é controlado com as teclas W, A, S e D.
            \par De modo a determinar a direção em que o Pacman se deve mover, o cliente tem de comparar as posições do Pacman e do ponteiro. De forma geral, o Pacman tenta deslocar-se no sentido que minimiza a sua distância ao ponteiro, ou seja, ao longo do eixo que apresenta a maior distância entre os dois. Por outras palavras, se o ponteiro estiver "mais para a direita" (face às outras direções) do Pacman, este move-se para a direita. Este processo é realizado pela função \textit{handle\_user\_input} de \textit{client.c}, o excerto relevante da qual está abaixo:
            
            \begin{lstlisting}
                static char last_move_pac = -1;
                static char curr_move_pac = -1;
                int mouse_x = 0, mouse_y = 0, board_x = 0, board_y = 0;
                // If the user is pressing LMB
                if (SDL_GetMouseState(&mouse_x, &mouse_y) & SDL_BUTTON(SDL_BUTTON_LEFT))
                {
                    get_board_place(mouse_x, mouse_y, &board_x, &board_y);
                    Player* player = player_find_by_id(game, game->player_id);
                    int pac_x = player_get_pac_pos_x(player), pac_y = player_get_pac_pos_y(player);
                    // Compute the distance between the pacman and the tile the mouse points to
                    // The pacman will move in the direction which shows the biggest discrepancy, or not move if null distance
                    if (board_x == pac_x && board_y == pac_y)                       // If the mouse is over the pacman
                    {
                        curr_move_pac = (char)0;
                    }
                    else if (abs_int(board_x - pac_x) > abs_int(board_y - pac_y))   // If further in the x direction
                    {
                        if (board_x > pac_x)
                            curr_move_pac = 'd';
                        else
                            curr_move_pac = 'a';
                    }
                    else                                                            // If further in the y direction
                    {
                        if (board_y > pac_y)
                            curr_move_pac = 's';
                        else
                            curr_move_pac = 'w';
                    }
                }
                else    // If the user isn't pressing mouse 1
                {
                    curr_move_pac = (char)0;
                }
            \end{lstlisting}
            
            \par O movimento do monstro é implementado de forma ligeiramente mais complexa. Regra geral, jogador espera poder segurar a tecla A para se mover para a esquerda, carregar na tecla W para se mover para cima e, ao largar o W, continuar a mover-se para a esquerda sem ter de pressionar a tecla A novamente. Para que isto aconteça, é implementado um \textit{stack} de 4 teclas, de modo a que, qualquer que seja a combinação de teclas premida, ao largar a última tecla, a penúltima (desde que continue premida) dita a direção de movimento. O excerto relevante (presente na mesma função que o excerto acima) está abaixo:
            \begin{lstlisting}
                // Stores the WASD keys in order of them being pressed
                // They are removed once released
                static char wasd_stack[4] = {0,0,0,0};
                // Stores which of they keys is currently pressed as a 1
                // W is idx 0, A is idx 1,...
                // WASD
                static unsigned int wasd_pressed[4] = {0,0,0,0};
                static unsigned int n_pressed = 0;
            
                const Uint8* keys_pressed = SDL_GetKeyboardState(NULL);
            
                // If the state of the a key differs from the last frame or in other words
                //  if the user pressed or released a key this frame
                if (keys_pressed[SDL_SCANCODE_W] != wasd_pressed[0])    // Update the stack and pressed arrays
                    update_key(wasd_stack, wasd_pressed, &n_pressed, 'w', keys_pressed[SDL_SCANCODE_W]);
            
                else if (keys_pressed[SDL_SCANCODE_A] != wasd_pressed[1])
                    update_key(wasd_stack, wasd_pressed, &n_pressed, 'a', keys_pressed[SDL_SCANCODE_A]);
            
                else if (keys_pressed[SDL_SCANCODE_S] != wasd_pressed[2])
                    update_key(wasd_stack, wasd_pressed, &n_pressed, 's', keys_pressed[SDL_SCANCODE_S]);
            
                else if (keys_pressed[SDL_SCANCODE_D] != wasd_pressed[3])
                    update_key(wasd_stack, wasd_pressed, &n_pressed, 'd', keys_pressed[SDL_SCANCODE_D]);
            
                // The monster's last and current movement directions
                static char last_move_mon = (char)-1;
                static char curr_move_mon = (char)-1;
                // Update the current movement direction
                if (n_pressed)                                      // If any keys are pressed, movement is the last pressed key
                    curr_move_mon =  wasd_stack[n_pressed - 1];
                else                                                // Otherwise the monster stops
                    curr_move_mon = (char)0;
                // If there is a new movement direction (different from the last one sent)
            
            \end{lstlisting}
            \par O stack em si é gerido pela função \textit{update\_key}. A função é simples, adicionando uma tecla ao \textit{stack} quando esta é premida, e removendo quando é solta. No entanto é algo extensa, pelo que será omitida do relatório. Econtra-se no mesmo ficheiro que a anterior.

        \subsection{Validação do número de jogadores}
            
            \par Devido ao fato do tabuleiro de jogo ser limitado, o servidor tem de estabelecer um limite de clientes. O número máximo de jogadores depende do número de células vazias do tabuleiro. O número de células \(C\) ocupadas por \(N\) jogadores vem:
            \begin{equation}
                C = 2N + 2 \textit{max}(0, N-1)
            \end{equation}
            \par Invertendo a equação, obtermos o número máximo de jogadores para um dado número de células como:
            \begin{equation}
                N_\textit{max} = \textit{floor}(\frac{C + 2}{4})
            \end{equation}

            \par O número máximo de jogadores é determinado na \textit{main} de \textit{server.c}:
            \begin{lstlisting}
                // Read the board
                unsigned int n_empty = read_board(game, "board-drawer/board.txt");

                // Calculate maximum number of clients
                // This magic formula apparently works, yay for no nested condition mess
                game->max_players = floor(((float)n_empty + 2.0) / 4.0);
            \end{lstlisting}
            \par A leitura do tabuleiro retorna o número de células vazias, que é utilizado para calcular o número máximo de jogadores. 
            \par A função que aceita as novas conexões verifica se o servidor está cheio antes de guardar a informação relevante ao novo cliente (excerto de \textit{connect\_to\_clients} em \textit{server\_connection.c}):
            \begin{lstlisting}
                // Accept connections - blocks until a client connects
                client_socket = accept(listen_socket, NULL, NULL);
                // ... 
                if (game_is_full(game))
                {
                    puts("Server is full, denying connection request");
                    message_send_server_full(client_socket);
                    continue;
                }
            \end{lstlisting}

        \subsection{Colocação de novos jogadores no tabuleiro}
        
            \par De modo a colocar os personagens em sítios aleatórios, recorre-se à função \textit{board\_random\_empty\_space} de \textit{pacman.c}:
            \begin{lstlisting}
                if (!board_empty_space_exists(board))   // This prevents the mythical infinite loop
                {
                    *x = -1; *y = -1;
                    return;
                }
                do
                {
                    *x = rand() % vec_get_x(board->board_size); // It aint efficient
                    *y = rand() % vec_get_y(board->board_size); //  but it works, it just works
                }
                while (board->board[*x][*y] != TILE_EMPTY);
            
            \end{lstlisting}
            \par Esta verifica primeiro se existe um espaço vazio, e depois gera coordenadas aleatórias até ser satisfeita a condição da célula se encontrar livre.
            \par Quando um cliente se conecta geram-se duas posições aleatórias, nas quais se colocam o Pacman e o monstro. O excerto seguinte é da função \textit{player\_create} de \textit{server.c}:
            \begin{lstlisting}
                // Put characters in a random (empty) position
                int x, y;
                board_random_empty_space(game->board, &x, &y);  // Pacman
                new_player->pacman_pos = vec_create(x, y);
                board_set_tile(game->board, x, y, board_player_id_to_tile_type(player_id, 1));
                board_random_empty_space(game->board, &x, &y);  // Monster
                new_player->monster_pos = vec_create(x, y);
                board_set_tile(game->board, x, y, board_player_id_to_tile_type(player_id, 0));
            \end{lstlisting}
            
        \subsection{Desconexão dos clientes}
            
            \par Quando um dos lados de um \textit{socket INET} fecha a sua socket, a função \textit{recv} retorna 0 no lado oposto. O servidor faz uso disto para determinar se um cliente se desconecta, como se pode observar no excerto seguinte, da função \textit{recv\_from\_client} em \textit{server\_connection.c}:
            \begin{lstlisting}
                // Determine message type
                MessageType mt = MESSAGE_TERMINATOR;
                int ret = message_recv_uint16_t(client->socket, (uint16_t*)&mt);
                if (ret == 0)
                {
                    fprintf(stdout, "Client %d left!\n", client->player_id);
                    break;
                }
            \end{lstlisting}
            \par Ao ler o tipo de mensagem enviado pelo cliente, verificando que o \textit{recv} retorna 0, o sevidor quebra o \textit{loop} e destrói as estruturas \textit{player} e \textit{client} associadas ao cliente, notificando os clientes restantes da desconexão.

        \subsection{Movimento dos personagens}
        
            \par Para a tornar mais breve, nesta secção será abordado apenas o movimento do Pacman. O movimento do monstro é implementado de forma análoga, com as regras diferentes estabelecidas no enunciado do projeto.
            \par A gestão do movimento do Pacman é feito pela função \textit{handle\_pacman\_move} em \textit{server.c}. Esta começa por verificar se a célula alvo (para a qual o Pacman tem o intuito de se mover) está \textit{out of bounds}, ou \textit{OOB}. Se for o caso, o movimento é feito considerando que o alvo é um tijolo, pois ambos os casos são funcionalmente idênticos.
            \begin{lstlisting}
                unsigned int tile = board_get_tile(game->board, tgt_x, tgt_y);
                if (board_is_oob(game->board, tgt_x, tgt_y))        // If the target is OOB (out of bounds)
                    tile = TILE_BRICK;                              // Handle collisions as if the target were a brick (bounce back)
                switch (tile)
                {
                // ...
            \end{lstlisting}
            \par Caso o alvo seja um tijolo (ou, como foi visto antes, \textit{OOB}), verifica-se se é possível mover para a célula oposta e, caso esteja, a função é chamada recursivamente com a célula oposta como alvo.
            \begin{lstlisting}
                case TILE_BRICK:            // Bounce back (if able)
                tgt_x = curr_x; tgt_y = curr_y;
                get_opposite_target_tile(&tgt_x, &tgt_y, player->pacman_move_dir);                                          // Invert the movement direction
                if (board_is_oob(game->board, tgt_x, tgt_y) || board_get_tile(game->board, tgt_x, tgt_y) == TILE_BRICK);    // If the new target tile is OOB or a brick
                    // Stay in place, do nothing
                else
                    handle_pacman_move(game, player, tgt_x, tgt_y);                                                         // Handle the new target tile
                break;
            \end{lstlisting}
            \par Caso a célula alvo esteja livre, o Pacman é simplesmente movido para a célula em questão. Existem duas representações da posição do Pacman, dentro da estrutura \textit{Player} e no tabuleiro, sendo que ambas têm de ser atualizadas.
            \par Quando um fruto se encontra presente na célula alvo, faz-se recurso à função \textit{handle\_fruit\_eat}, que trata de mover o Pacman, incrementar a sua pontuação, torná-lo no \textit{powered-up} Pacman, e eliminar o fruto (temporariamente) do tabuleiro.
            \par A troca de posição é algo trivial, resumindo-se à troca dos vetores de posição dos dois personagens, e das células respetivas no tabuleiro.
            \par Quanto às interações com monstros inimigos, utiliza-se a função \textit{handle\_character\_eat}, que processa um personagem a comer outro, tendo em conta qual dos dois se move. Neste caso o Pacman é o personagem que se move. Se estiver também \textit{powered-up}, este move-se para a célula do monstro, tem o seu \textit{score} incrementado, estado de \textit{power-up} decrementado, e o monstro é movido para uma posição livre aleatória. Caso contrário, o monstro permanece na mesma posição, o seu jogador tem a pontuação incrementada, e o Pacman desloca-se para uma posição aleatória.

        \subsection{Temporização}
            \par Existem três eventos de jogo que dependem de do tempo: O movimento dos personagens, que ocorre a cada 0.5 segundos, o \textit{respawn} das frutas, que ocorre passado 2 segundos de terem sido comidas, e o movimento aleatório dos personagens por inatividade, que ocorre passado 30 segundos do último movimento do mesmo.
            \par Todos estes eventos são geridos com o auxílio das funções \textit{clock\_gettime} (de \textit{time.h}), para guardar o tempo em que determinado evento ocorreu e determinar o tempo atual, e \textit{time\_diff\_ms} (em \textit{utilities.c}), que retorna a diferença entre dois instantes definidos por \textit{struct timespec} (o formato de tempo utilizado por \textit{clock\_gettime}).
            \par Tomando o exemplo de quando um Pacman come uma fruta:
            \begin{lstlisting}
                case TILE_FRUIT:            // Eat the fruit
                {
                    Fruit* fruit = fruit_find_by_pos(game, tgt_x, tgt_y);
                    handle_fruit_eat(game, fruit, player, 1);
                    clock_gettime(CLOCK_MONOTONIC, &player->pacman_last_move_time);    
                    break;
                }
            \end{lstlisting}
            \par Como se observa, utiliza-se a função \textit{clock\_gettime} para guardar o instante em que o Pacman se move. A função \textit{handle\_fruit\_eat} também guarda o instante em que a fruta foi comida.
            \par De modo a mover os personagens só após 0.5 segundos, e movê-los passado os 30 segundos de inatividade, a função \textit{game\_update} (em \textit{server.c}) avalia a diferença entre o instante atual e o instante em que o personagem se moveu por último:
            \begin{lstlisting}
                // Move the characters if enough time has passed
                struct timespec now;
                clock_gettime(CLOCK_MONOTONIC, &now);
                for (unsigned int i = 0; i < game->n_players; ++i)
                {
                    Player* player = game->players[i];
                    // ...
                    if (time_diff_ms(player->pacman_last_move_time, now) > 500 && player->pacman_move_dir != (char)0)   // Same as above with pacman
                    {
                        int tgt_x = vec_get_x(player->pacman_pos), tgt_y = vec_get_y(player->pacman_pos);
                        get_target_tile(&tgt_x, &tgt_y, player->pacman_move_dir);
            
                        handle_pacman_move(game, player, tgt_x, tgt_y);
                    }
                    else if (time_diff_ms(player->pacman_last_move_time, now) > 30000)
                    {
                        handle_inactivity_move(game, player, 1);
                    }
                    // ...
                }
            \end{lstlisting}
            \par A mesma função trata de colocar os frutos no tabuleiro se tiverem sido comidos à mais de 2 segundos:
            \begin{lstlisting}
                // Respawn the fruits if enough time has passed
                for (unsigned int i = 0; i < game->n_fruits; ++i)
                {
                    Fruit* fruit = game->fruits[i];
                    if (!fruit->is_alive && time_diff_ms(fruit->eaten_time, now) > 2000)
                        fruit_respawn(game, fruit);
                }            
            \end{lstlisting}

        \subsection{\textit{Power-up} do Pacman}
            \par O estado de \textit{power-up} do Pacman, como foi visto anteriormente, está simplesmente definido como um inteiro na estrutura \textit{Player}. Quando o Pacman come um fruto guarda-se o número 2 na variável, sendo esta decrementada quando come um monstro. Desta forma, quando o Pacman come dois monstros a variável chega a 0, e o Pacman deixa de estar \textit{powered-up}.

        \subsection{Envio da tabela de pontuação}
            \par A tabela de pontuação (ou, mais corretamente, a ordem para a imprimir na consola) é enviada a cada 60 segundos para todos os clientes a cada 60 segundos. A secção responsável da \textit{main} do servidor está abaixo:
            \begin{lstlisting}
                // Update the clients with information (if there are any)
                if (game->n_players)
                {
                    // ...
                    // Scoreboard is only sent every 60s
                    static struct timespec last_scoreboard_time;
                    if (last_scoreboard_time.tv_sec == 0)   // Static variables are thankfuly init'd as 0
                        clock_gettime(CLOCK_MONOTONIC, &last_scoreboard_time);
                    struct timespec now;
                    clock_gettime(CLOCK_MONOTONIC, &now);
                    if (time_diff_ms(last_scoreboard_time, now) > 60000)
                    {
                        send_to_all_clients(game, MESSAGE_PRINT_SCOREBOARD, NULL);
                        clock_gettime(CLOCK_MONOTONIC, &last_scoreboard_time);
                    }
                }
            \end{lstlisting}

        \subsection{Limpeza de dados e libertação de recursos}
            \par A limpeza de dados é realizada pelas funções marcadas com \textit{destroy}. Por exemplo, a função \textit{player\_destroy} (em \textit{server.c}) remove um \textit{Player} do array do servidor, libertando toda a memória utilizada por este. Estão ainda definidas funções equivalentes para destruir frutas, clientes (a estrutura de \textit{server\_connection.c}) e vetores.

    \section{Sincronização}
        \par O servidor faz uso de várias \textit{threads} que trabalham com o mesmo conjunto de dados. Por exemplo, o \textit{array} de \textit{Clients} é manuseado pela função que aceita novas conexões, por cada função que recebe informação de um cliente, e pela função que envia uma determinada mensagem a todos os clientes. Ora se a primeira juntasse um novo cliente ao \textit{array}, ou uma das segundas o removesse, enquanto a última estava a meio de um envio, poderia ocorrer um acesso de memória ilegal.
        \par Deste modo, estabelece-se um \textit{mutex}, \textit{client\_array\_lock}, que é \textit{locked} e \textit{unlocked} no decorrer das funções \textit{client\_store}, \textit{client\_destroy} e \textit{send\_to\_all\_clients}, de forma a que não sejam executadas simultâneamente.
        \par O mesmo acontece com o array de \textit{Players}. As funções \textit{player\_create} e \textit{player\_destroy} fazem uso do mesmo sistema com o seu próprio \textit{mutex}, \textit{player\_array\_lock}.
\end{document}